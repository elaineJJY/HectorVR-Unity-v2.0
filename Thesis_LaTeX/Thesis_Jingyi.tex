% !TeX program = lualatex

\documentclass[
	ngerman,
	ruledheaders=section,%Ebene bis zu der die Überschriften mit Linien abgetrennt werden, vgl. DEMO-TUDaPub
	class=report,% Basisdokumentenklasse. Wählt die Korrespondierende KOMA-Script Klasse
	thesis={type=bachelor},% Dokumententyp Thesis, für Dissertationen siehe die Demo-Datei DEMO-TUDaPhd
	accentcolor=9c,% Auswahl der Akzentfarbe
	custommargins=false,% Ränder werden mithilfe von typearea automatisch berechnet
	marginpar=false,% Kopfzeile und Fußzeile erstrecken sich nicht über die Randnotizspalte
	%BCOR=5mm,%Bindekorrektur, falls notwendig
	parskip=half-,%Absatzkennzeichnung durch Abstand vgl. KOMA-Script
	fontsize=11pt,%Basisschriftgröße laut Corporate Design ist mit 9pt häufig zu klein
%	logofile=example-image, %Falls die Logo Dateien nicht vorliegen
]{tudapub}


% Der folgende Block ist nur bei pdfTeX auf Versionen vor April 2018 notwendig
\usepackage{iftex}
\ifPDFTeX
	\usepackage[utf8]{inputenc}%kompatibilität mit TeX Versionen vor April 2018
\fi

%%%%%%%%%%%%%%%%%%%
%Sprachanpassung & Verbesserte Trennregeln
%%%%%%%%%%%%%%%%%%%
\usepackage[ngerman, main=english]{babel}
\usepackage[autostyle]{csquotes}% Anführungszeichen vereinfacht
\usepackage{microtype}

%%%%%%%%%%%%%%%%%%%
%Literaturverzeichnis
%%%%%%%%%%%%%%%%%%%
%\usepackage{biblatex}   % Literaturverzeichnis
\usepackage[backend=biber,style=numeric,sorting=none]{biblatex}

\addbibresource{BibTexDatei.bib}

%%%%%%%%%%%%%%%%%%%
%Paketvorschläge Tabellen
%%%%%%%%%%%%%%%%%%%
%\usepackage{array}     % Basispaket für Tabellenkonfiguration, wird von den folgenden automatisch geladen
\usepackage{tabularx}   % Tabellen, die sich automatisch der Breite anpassen
%\usepackage{longtable} % Mehrseitige Tabellen
%\usepackage{xltabular} % Mehrseitige Tabellen mit anpassarer Breite
\usepackage{booktabs}   % Verbesserte Möglichkeiten für Tabellenlayout über horizontale Linien

%%%%%%%%%%%%%%%%%%%
%Paketvorschläge Mathematik
%%%%%%%%%%%%%%%%%%%
%\usepackage{mathtools} % erweiterte Fassung von amsmath
%\usepackage{amssymb}   % erweiterter Zeichensatz
%\usepackage{siunitx}   % Einheiten

%Formatierungen für Beispiele in diesem Dokument. Im Allgemeinen nicht notwendig!
\let\file\texttt
\let\code\texttt
\let\tbs\textbackslash

\usepackage{pifont}% Zapf-Dingbats Symbole
\newcommand*{\FeatureTrue}{\ding{52}}
\newcommand*{\FeatureFalse}{\ding{56}}

\begin{document}

\Metadata{
	title= Concepts for operating ground based rescue robots using virtual reality,
	author=Jingyi Jia
}

\title{Concepts for operating ground based rescue robots using virtual reality}
%\subtitle{\LaTeX{} using TU Darmstadt's Corporate Design}
\author{Jingyi Jia}%optionales Argument ist die Signatur,
\birthplace{Shanghai, China}%Geburtsort, bei Dissertationen zwingend notwendig
\reviewer{Prof. Dr. Max Mühlhäuser \and Julius von Willich}%Gutachter


\department{inf} % Das Kürzel wird automatisch ersetzt und als Studienfach gewählt, siehe Liste der Kürzel im Dokument.
\institute{Telekooperation}
\group{Prof. Dr. Max Mühlhäuser}

\submissiondate{\today}
\examdate{\today}


\maketitle
\affidavit
\tableofcontents

\selectlanguage{english}
\begin{abstract}
Rescue robotics have been increasingly used in crisis situations, such as exploring areas that are too dangerous for humans, and have become a key research area. A number of studies use \gls{vr} to improve the degree of immersion and situation awareness. However, there remains a need to explore an intuitive, easy-to-use interaction pattern, which increases the efficiency of search and rescue and allows the user to explore the environment intentionally. This thesis presents a preliminary VR-based \gls{hri} system in terms of ground based rescue robots, with the aim to find an ideal interaction pattern. The proposed system offers four different operation modes and corresponding test scenes imitating a post-disaster city. This thesis includes a user study in which four different operation modes are tested in turn. The user study reveals that the ideal interaction pattern should reduce the complexity of the operation as much as possible. Instead of allowing the user to adjust the robot's direction and speed themselves, it is recommended to set a target point and let the robot navigate to the target point automatically. 
\end{abstract}



\selectlanguage{ngerman}
\begin{abstract}
Rettungsroboter werden zunehmend zur Bewältigung von Krisensituationen eingesetzt z.B. dort, wo es für den Menschen zu gefährlich ist. Eine Reihe von Studien nutzt \gls{vr} als Plattform für die Mensch-Roboter-Interaktion, da dies die Immersion erhöhen und das Situationsbewusstsein verbessern kann. Es besteht jedoch nach wie vor Bedarf für eine einfach zu bedienende Interaktionsmethode, um das Personal bei der Steuerung des Roboters zu entlasten und die Effizienz und Effektivität von Such- und Rettungsarbeiten zu erhöhen. In dieser Abschlussarbeit wird ein vorläufiges \gls{vr}-basiertes Mensch-Roboter-Interaktionssystem in Bezug auf bodenbasierte Rettungsroboter vorgestellt. Ziel ist, ein möglichst intuitive Steuerungsmethode mit geringer mentaler Ermüdung zu finden. Das vorgeschlagene System bietet vier verschiedene Betriebsmodi und entsprechende Testszenen. Diese Arbeit beinhaltet eine Nutzerstudie (die vier Betriebsmodi werden nacheinander getestet und verglichen). Das Fazit ist, dass die ideale Interaktionsmethode die Komplexität der Steuerung so weit wie möglich zu reduzieren.
\end{abstract}

\selectlanguage{english}

\chapter{Introduction}

In recent years, natural disasters such as earthquakes, tsunamis and potential nuclear, chemical, biological and explosives have seriously threatened the safety of human life and property. While the number of various disasters has increased, their severity, diversity and complexity have also gradually increased. The 72h after a disaster is the golden rescue time, but the unstructured environment of the disaster site makes it difficult for rescuers to work quickly, efficiently and safely.

Rescue robots have the advantages of high mobility and handling breaking capacity, can work continuously to improve the efficiency of search and rescue, and can achieve the detection of graph, sound, gas and temperature within the ruins by carrying a variety of sensors, etc.
Moreover, the robot rescue can assist or replace the rescuers to avoid the injuries caused by the secondary collapse and reduce the risk of rescuers. Therefore, rescue robots have become an important development direction.

In fact, rescue robots have been put to use in a number of disaster scenarios. The Center for Robot-Assisted Search and Rescue (CRASAR) used rescue robots for Urban Search and Rescue (USAR) task during the World Trade Center collapse in 2001 \cite{Casper:2003tk} and has employed rescue robots at multiple disaster sites in the years since to assist in finding survivors, inspecting buildings and scouting the site environment etc \cite{Murphy:2012th}. Anchor Diver III was utilized as underwater support to search for bodies drowned at sea after the 2011 Tohoku Earthquake and Tsunami \cite{Huang:2011wq}.

Considering the training time and space constraints for rescuers \cite{Murphy:2004wl}, and the goal of efficiency and fluency collaboration \cite{10.1145/1228716.1228718}, the appropriate human-robot interaction approach deserves to be investigated. Some of the existing human-computer interaction methods are Android software \cite{Sarkar:2017tt} \cite{Faisal:2019uu}, gesture recognition\cite{Sousa:2017tn} \cite{10.1145/2157689.2157818} \cite{Nagi:2014vu}, facial voice recognition \cite{Pourmehr:2013ta}, adopting eye movements \cite{Ma:2015wu}, Augmented Reality(AR)\cite{SOARES20151656} and Virtual Reality(VR), etc.

% VR and robot
Among them, VR has gained a lot of attention due to its immersion and the interaction method that can be changed virtually. VR is no longer a new word. With the development of technology in recent years, VR devices are gradually becoming more accessible to users. With the improvement of hardware devices, the new generation of VR headsets have higher resolution and wider field of view. And in terms of handle positioning, with the development of computer vision in the past few years, VR devices can now use only the four cameras mounted on the VR headset to achieve accurate spatial positioning, and support hand tracking, accurately capturing every movement of hand joints. While VR are often considered entertainment devices, VR brings more than that. It plays an important role in many fields such as entertainment, training, education and medical care.

The use of VR in human-computer collaboration also has the potential. In terms of reliability, VR is reliable as a novel alternative to human-robot interaction. The interaction tasks that users can accomplish with VR devices do not differ significantly from those using real operating systems\cite{Villani:2018ub}. In terms of user experience and operational efficiency, VR displays can provide users with stereo viewing cues, which makes collaborative human-robot interaction tasks in certain situations more efficient and performance better \cite{Liu:2017tw}.

However, there remains a need to explore human-computer interaction patterns and improve the level of human-computer integration\cite{Wang:2017uy}. Intuitive and easy-to-use interaction methods can enable the user to explore the environment as intentionally as possible and improve the efficiency of search and rescue.

% What I have done (overview)
For this purpose, this paper presents a preliminary VR-based system for the simulation of ground rescue robots with four different modes of operation and corresponding test scenes imitating a post-disaster city. The test scene simulates a robot collaborating with Unity to construct a virtual 3D scene. The robot has a simulated radar, which makes the display of the scene dependent on the robot's movement.
In order to find a control method that is as intuitive and low mental fatigue as possible, a user survey was executed after the development was completed.



% Paper Architecture
Section \ref{implementation} provides details of the purposed system, including the techniques used for the different interaction modes and the structure of the test scenarios.
Section \ref{evaluate} will talk about the design and process of user study.
Section \ref{result} presents the results of the user study and analyzes the advantages and disadvantages of the different modes of operation and the directions for improvement.
Finally, in Section \ref{conclusion}, conclusions and future work are summarized.



\chapter{Related Work}
\label{related}

In this chapter, some research on the integration of \gls{vr} and \gls{hri} will be discussed. The relevant literature and its contributions will be briefly presented. 

The topic of \gls{vr} and \gls{hri} is an open research topic with many kinds of focus perspectives.

\gls{hri} platforms combined with virtual worlds have many applications. It can be used, for example, to train machine operators in factories. Elias Matsas et al. \cite{Matsas:2017aa} provided a \gls{vr}-based training system using hand recognition. Kinect cameras are used to capture the user's positions and motions, and virtual user models are constructed in the \gls{vr} environment based on the collected data. Users will operate robots and virtual objects in the \gls{vr} environment, and in this way, learn how to operate the real robot. The framework proposed by Luis Pérez et al. \cite{Perez:2019ub} is applied to train operators to learn to control the robot. Since the environment does not need to change in real time, but rather needs to recreate the factory scene realistically, a highly accurate 3D environment was constructed in advance using Blender in combination with a 3D scanner.

Building 3D scenes in virtual worlds based on information collected by robots is also a research highlight. Wang, et al. \cite{Wang:2017uy} were concerned with the visualization of the rescue robot and its surroundings in a virtual environment. The proposed \gls{hri} system uses the incremental 3D-NDT map to render the robot's surroundings in real time. The user can view the robot's surroundings in a first-person view through the \gls{htc} and send control commands through arrow keys on the motion controllers. A novel \gls{vr}-based practical system is presented in \cite{Stotko:2019ud} consisting of distributed systems to reconstruct the 3D scene. The data collected by the robot is first transmitted to the client responsible for reconstructing the scene. After the client has constructed the 3D scene, the set of actively reconstructed visible voxel blocks is sent to the server responsible for communication, which has a robot-based live telepresence and teleoperation system. This server will then broadcast the data back to the client used by the operator, thus enabling an immersive visualization of the robot within the scene.

Others are more concerned about the manipulation of the robotic arm mounted on the robot. Moniri et al. \cite{Moniri:2016ud} provided a \gls{vr}-based operating model for the robotic arm. The user wearing a headset can see a simulated 3D scene at the robot's end and send pickup commands to the remote robot by clicking on the target object with the mouse. The system proposed by Ostanin et al. \cite{Ostanin:2020uo} is also worth mentioning. Although their proposed system for operating a robotic arm is based on \gls{mr}, the article is highly relevant to this thesis, considering the correlation of \gls{mr} and \gls{vr} and the proposed system detailing the combination of \gls{ros} and robotics. In their system, \gls{ros} Kinect is used as middleware and responsible for communicating with the robot and the \gls{unity} side. The user can control the movement of the robot arm by selecting predefined options in the menu. In addition, the orbit and target points of the robot arm can be set by clicking on a hologram with a series of control points.

%Summary
To summarize, previous work has studied methods and tools for \gls{vr}-based \gls{hri} and teleoperation. However, only few studies focus on the different interactive approaches for \gls{hri}.
\chapter{Implementation}
\label{implementation}

% summary
In this chapter, the tools and techniques used in building this human-computer collaborative VR-based system are described. The focus will be on interaction techniques for different modes of operation. In addition, the construction of test scenarios and the setup of the robot will also be covered in this chapter.


\section{Overview}
The main goal of this work is to design and implement a VR-based human-robot collaboration system with different methods of operating the robot in order to find out which method of operation is more suitable to be used to control the rescue robot. Further, it is to provide some basic insights for future development directions and to provide a general direction for finding an intuitive, easy-to-use and efficient operation method. Therefore, the proposed system was developed using Unity, including four modes of operation and a corresponding test environment for simulating post-disaster scenarios. In each operation mode, the user has a different method to control the robot. In addition, in order to better simulate the process by which the robot scans its surroundings and the computer side cumulatively gets a reconstructed 3D virtual scene, the test environment was implemented in such a way that the picture seen by the user depends on the direction of the robot's movement and the trajectory it travels through.


\section{System Architecture}
The proposed system runs on a computer with the Windows 10 operating system. This computer has been equipped with an Intel Core i7-8700K CPU, 32 GB RAM as well as a NVIDIA GTX 1080 GPU with 8 GB VRAM. HTC Vive is used as a VR device. It has a resolution of 1080 × 1200 per eye, resulting in a total resolution of 2160 × 1200 pixels, a refresh rate of 90 Hz, and a field of view of 110 degrees. It includes two motion controllers and uses two Lighthouses to track the position of the headset as well as the motion controllers.

Unity was chosen as the platform to develop the system. Unity is a widely used game engine with Steam VR plugin \footnote{https://assetstore.unity.com/packages/tools/integration/steamvr-plugin-32647}, which allows developers to focus on the VR environment and interactive behaviors in programming, rather than specific controller buttons and headset positioning, making VR development much simpler. Another reason why Unity was chosen as a development platform is the potential for collaboration with the Robot Operating System (ROS), a frequently used operating system for robot simulation and manipulation, which is flexible, low-coupling, distributed, open source, and has a powerful and rich third-party feature set. In terms of collaboration between Unity and ROS, Siemens provides open source software libraries and tools in C\# for communicating with ROS from .NET applications \footnote{https://github.com/siemens/ros-sharp}. Combining ROS and Unity to develop a collaborative human-robot interaction platform proved to be feasible \cite{Whitney:2018wk}. Since the focus of this paper is on human-robot interaction, collaboration and synchronization of ROS will not be explored in detail here.


\section{Interaction techniques}
This system has 4 different approaches to control the robot. Each mode has its own distinctive features: 

\begin{enumerate}
\item In Handle Mode the user will send control commands directly using the motion controller. 
\item In Lab Mode a simulated lab is constructed in the VR environment and the user will use virtual buttons in the lab to control the rescue robot. 
\item In Remote Mode the user can set the driving destination directly. 
\item In UI Mode the user has a virtual menu and sends commands via rays from the motion controller.
\end{enumerate}

In order to improve the reusability of the code and to facilitate the management of subsequent development, the classes that manage the interaction actions of each mode implement the same interface. A graphical representation of the system activities workflow is given in the UML activity diagram in Fig.\ref{fig:uml}.

\begin{figure}[h]
    \centering
    \includegraphics[height=14cm]{graphics/uml.png}
    \caption{UML Class diagram for the structure of the system}
    \label{fig:uml}
\end{figure}

\subsection{Handle Mode}
\subsection{Lab Mode}
\subsection{Remote Mode}
\subsection{UI Mode}


\section{Test Scene}
\chapter{Evaluation of User Experience}
\label{evaluate}


This chapter describes the design and detailed process of the user evaluation. The purpose of this user study is to measure the impact of four different modes of operation on rescue efficiency, robot driving performance, and psychological and physiological stress and fatigue, etc. For this purpose, participants are asked to find victims in a test scene using different modes of operation and to answer a questionnaire after the test corresponding to each mode of operation.



\section{Study Design}

The evaluation for each mode of operation consists of two main parts. The first part is the data recorded during the process of the participant driving the robot in the VR environment to find the victims. The recorded data includes information about the robot's collision and the speed of driving etc. The rescue of the victims was also considered as part of the evaluation. Besides the number of victims rescued, the number of victims who were already visible but neglected is also important data. The Official NASA Task Load Index (TLX) was used to measure the participants subjective workload asessments. Additionally, participants were asked specific questions for each mode of operation and were asked to select their favorite and least favorite modes of operation.



\section{Procedure}
Before the beginning of the actual testing process, participants were informed of the purpose of the project, the broad process and the content of the data that would be collected. After filling in the basic demographics, the features of each of the four modes of operation and their rough usage were introduced verbally with a display of the buttons on the motion controllers.



\section{Entering the world of VR}
After the basic introduction part, participants would directly put on the VR headset and enter the VR environment to complete the rest of the tutorial. Considering that participants might not have experience with VR and that it would take time to learn how to operate the four different modes, the proposed system additionally sets up a practice pattern and places some models of victims in the practice scene. After entering the VR world, participants first needed to familiarize themselves with the opening and selecting options of the menu, as this involves switching between different modes and entering the test scenes. Then participants would use the motion controllers to try to teleport themselves, or raise themselves into mid-air. Finally participants were asked to interact with the victim model through virtual hands. After this series of general tutorials, participants were already generally familiar with the use of VR and how to move around in the VR world.



\section{Practice and evaluation of modes}
Given the different manipulation approaches for each mode, in order to avoid confusion between the different modes, participants would then take turns practicing and directly evaluating each mode immediately afterwards. The participant first switched to the mode of operation to be tested and manipulated the robot to move in that mode. After attempting to rescue 1-2 victim models and the participant indicated that he or she was familiar enough with this operation mode, the participant would enter the test scene. In the test scene, participants had to save as many victims as possible in a given time limit. Participants were required to move the robot around the test scene to explore the post-disaster city and to find and rescue victims. In this process, if the robot crashes with buildings, obstacles, etc., besides the collision information being recorded as test data, participants would also receive sound and vibration feedback. The test will automatically end when time runs out or when all the victims in the scene have been rescued. Participants were required to complete the evaluation questionnaire and the NASA evaluation form at the end of each test. This process was repeated in each mode of operation. 

After all the tests were completed, participants were asked to compare the four operation modes and select the one they liked the most and the one they liked the least. In addition, participants could give their reasons for the choice and express their opinions as much as they wanted, such as suggestions for improvement or problems found during operation.

\chapter{Conclusion}
\label{conclusion}

This thesis presents a preliminary VR-based \gls{hri} system in terms of ground based rescue robots. This work aims to find an ideal interaction method or provide a general direction for future development. For this purpose, the proposed system offers four different operation modes and corresponding test scenes imitating a post-disaster city. This thesis shows an overview of the simulated robot, interaction techniques and the construction of the test environment. Eight participants were invited to conduct a user study. Based on the obtained results, it can be concluded that an ideal \gls{vr}-based robotics operation method should eliminate as much complexity as possible. An intelligent obstacle avoidance algorithm is recommended instead of the user operating the robot themselves to steer and move forward. Additional functions, such as monitoring screens, need to be optimized so that they do not complicate the whole interaction process. The system also requires maps to show the user which areas have been detected and where the robot is located.

Future work should focus on the intelligent obstacle avoidance algorithm when using a real robot. The next stage is to develop a live telepresence and teleoperation system with the real robot. Considering that the system proposed in this thesis only simulates the disaster rescue process, the conclusions obtained may be limited. Additional testing and user surveys should be carried out in the future after building a collaborative \gls{vr}-based system with real robots.




\printbibliography

\end{document}
