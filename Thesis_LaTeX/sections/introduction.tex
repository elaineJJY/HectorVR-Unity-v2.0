\chapter{Introduction}
In recent years, natural disasters such as earthquakes, tsunamis and potential nuclear, chemical, biological and explosives have seriously threatened the safety of human life and property. While the number of various disasters has increased, their severity, diversity and complexity have also gradually increased. The 72h after a disaster is the golden rescue time, but the unstructured environment of the disaster site makes it difficult for rescuers to work quickly, efficiently and safely.

Rescue robots have the advantages of high mobility and handling breaking capacity, can work continuously to improve the efficiency of search and rescue, and can achieve the detection of graph, sound, gas and temperature within the ruins by carrying a variety of sensors, etc.
Moreover, the robot rescue can assist or replace the rescuers to avoid the injuries caused by the secondary collapse and reduce the risk of rescuers. Therefore, rescue robots have become an important development direction.

In fact, rescue robots have been put to use in a number of disaster scenarios. The Center for Robot-Assisted Search and Rescue (CRASAR) used rescue robots for Urban Search and Rescue (USAR) task during the World Trade Center collapse in 2001 \cite{Casper:2003tk} and has employed rescue robots at multiple disaster sites in the years since to assist in finding survivors, inspecting buildings and scouting the site environment etc \cite{Murphy:2012th}. Anchor Diver III was utilized as underwater support to search for bodies drowned at sea after the 2011 Tohoku Earthquake and Tsunami \cite{Huang:2011wq}.

Considering the training time and space constraints for rescuers \cite{Murphy:2004wl}, and the goal of efficiency and fluency collaboration \cite{10.1145/1228716.1228718}, the appropriate human-robot interaction approach deserves to be investigated. Some of the existing human-computer interaction methods are Android software \cite{Sarkar:2017tt} \cite{Faisal:2019uu}, gesture recognition\cite{Sousa:2017tn} \cite{10.1145/2157689.2157818} \cite{Nagi:2014vu}, facial voice recognition \cite{Pourmehr:2013ta}, adopting eye movements \cite{Ma:2015wu}, Augmented Reality(AR)\cite{SOARES20151656} and Virtual Reality(VR), etc.