\chapter{Introduction}

In recent years, natural disasters such as earthquakes, tsunamis and potential nuclear, chemical, biological and explosives have seriously threatened the safety of human life and property. While the number of various disasters has increased, their severity, diversity and complexity have also gradually increased. The 72h after a disaster is the golden rescue time, but the unstructured environment of the disaster site makes it difficult for rescuers to work quickly, efficiently and safely.

Rescue robots have the advantages of high mobility and handling breaking capacity, can work continuously to improve the efficiency of search and rescue, and can achieve the detection of graph, sound, gas and temperature within the ruins by carrying a variety of sensors, etc.
Moreover, the robot rescue can assist or replace the rescuers to avoid the injuries caused by the secondary collapse and reduce the risk of rescuers. Therefore, rescue robots have become an important development direction.

In fact, rescue robots have been put to use in a number of disaster scenarios. The Center for Robot-Assisted Search and Rescue (CRASAR) used rescue robots for Urban Search and Rescue (USAR) task during the World Trade Center collapse in 2001 \cite{Casper:2003tk} and has employed rescue robots at multiple disaster sites in the years since to assist in finding survivors, inspecting buildings and scouting the site environment etc \cite{Murphy:2012th}. Anchor Diver III was utilized as underwater support to search for bodies drowned at sea after the 2011 Tohoku Earthquake and Tsunami \cite{Huang:2011wq}.

Considering the training time and space constraints for rescuers \cite{Murphy:2004wl}, and the goal of efficiency and fluency collaboration \cite{10.1145/1228716.1228718}, the appropriate human-robot interaction approach deserves to be investigated. Some of the existing human-computer interaction methods are Android software \cite{Sarkar:2017tt} \cite{Faisal:2019uu}, gesture recognition\cite{Sousa:2017tn} \cite{10.1145/2157689.2157818} \cite{Nagi:2014vu}, facial voice recognition \cite{Pourmehr:2013ta}, adopting eye movements \cite{Ma:2015wu}, Augmented Reality(AR)\cite{SOARES20151656} and Virtual Reality(VR), etc.

% VR and robot
Among them, VR has gained a lot of attention due to its immersion and the interaction method that can be changed virtually. VR is no longer a new word. With the development of technology in recent years, VR devices are gradually becoming more accessible to users. With the improvement of hardware devices, the new generation of VR headsets have higher resolution and wider field of view. And in terms of handle positioning, with the development of computer vision in the past few years, VR devices can now use only the four cameras mounted on the VR headset to achieve accurate spatial positioning, and support hand tracking, accurately capturing every movement of hand joints. While VR are often considered entertainment devices, VR brings more than that. It plays an important role in many fields such as entertainment, training, education and medical care.

The use of VR in human-computer collaboration also has the potential. In terms of reliability, VR is reliable as a novel alternative to human-robot interaction. The interaction tasks that users can accomplish with VR devices do not differ significantly from those using real operating systems\cite{Villani:2018ub}. In terms of user experience and operational efficiency, VR displays can provide users with stereo viewing cues, which makes collaborative human-robot interaction tasks in certain situations more efficient and performance better \cite{Liu:2017tw}.

However, there remains a need to explore human-computer interaction patterns and improve the level of human-computer integration\cite{Wang:2017uy}. Intuitive and easy-to-use interaction patterns can enable the user to explore the environment as intentionally as possible and improve the efficiency of search and rescue. The appropriate interaction method should cause less mental and physical exhaustion, which also extends the length of an operation, making it less necessary for the user to frequently exit the VR environment for rest.

% What I have done (overview)
For this purpose, this paper presents a preliminary VR-based system for the simulation of ground rescue robots with four different modes of operation and corresponding test scenes imitating a post-disaster city. The test scene simulates a robot collaborating with Unity to construct a virtual 3D scene. The robot has a simulated radar, which makes the display of the scene dependent on the robot's movement.
In order to find an interaction approach that is as intuitive and low mental fatigue as possible, a user study was executed after the development was completed.



% Paper Architecture
Chapter \ref{implementation} provides details of the purposed system, including the techniques used for the different interaction modes and the structure of the test scenes.
Chapter \ref{evaluate} will talk about the design and process of user study.
Chapter \ref{result} presents the results of the user study and analyzes the advantages and disadvantages of the different modes of operation and the directions for improvement.
Finally, in Chapter \ref{conclusion}, conclusions and future work are summarized.


