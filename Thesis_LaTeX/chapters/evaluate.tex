\chapter{Evaluation of User Experience}
\label{evaluate}


This chapter describes the design and detailed process of the user evaluation. The purpose of this user study is to measure the impact of four different modes of operation on rescue efficiency, robot driving performance, and psychological and physiological stress and fatigue, etc. For this purpose, participants are asked to find victims in a test scene using different modes of operation and to answer a questionnaire after the test corresponding to each mode of operation.



\section{Study Design}

The evaluation for each mode of operation consists of two main parts. The first part is the data recorded during the process of the participant driving the robot in the VR environment to find the victims. The recorded data includes information about the robot's collision and the speed of driving etc. The rescue of the victims was also considered as part of the evaluation. Besides the number of victims rescued, the number of victims who were already visible but neglected is also important data. The Official NASA Task Load Index (TLX) was used to measure the participants subjective workload asessments. Additionally, participants were asked specific questions for each mode of operation and were asked to select their favorite and least favorite modes of operation.



\section{Procedure}
Before the beginning of the actual testing process, participants were informed of the purpose of the project, the broad process and the content of the data that would be collected. After filling in the basic demographics, the features of each of the four modes of operation and their rough usage were introduced verbally with a display of the buttons on the motion controllers.



\section{Entering the world of VR}
After the basic introduction part, participants would directly put on the VR headset and enter the VR environment to complete the rest of the tutorial. Considering that participants might not have experience with VR and that it would take time to learn how to operate the four different modes, the proposed system additionally sets up a practice pattern and places some models of victims in the practice scene. After entering the VR world, participants first needed to familiarize themselves with the opening and selecting options of the menu, as this involves switching between different modes and entering the test scenes. Then participants would use the motion controllers to try to teleport themselves, or raise themselves into mid-air. Finally participants were asked to interact with the victim model through virtual hands. After this series of general tutorials, participants were already generally familiar with the use of VR and how to move around in the VR world.



\section{Practice and evaluation of modes}
Given the different manipulation approaches for each mode, in order to avoid confusion between the different modes, participants would then take turns practicing and directly evaluating each mode immediately afterwards. The participant first switched to the mode of operation to be tested and manipulated the robot to move in that mode. After attempting to rescue 1-2 victim models and the participant indicated that he or she was familiar enough with this operation mode, the participant would enter the test scene. In the test scene, participants had to save as many victims as possible in a given time limit. Participants were required to move the robot around the test scene to explore the post-disaster city and to find and rescue victims. In this process, if the robot crashes with buildings, obstacles, etc., besides the collision information being recorded as test data, participants would also receive sound and vibration feedback. The test will automatically end when time runs out or when all the victims in the scene have been rescued. Participants were required to complete the evaluation questionnaire and the NASA evaluation form at the end of each test. This process was repeated in each mode of operation. 

After all the tests were completed, participants were asked to compare the four operation modes and select the one they liked the most and the one they liked the least. In addition, participants could give their reasons for the choice and express their opinions as much as they wanted, such as suggestions for improvement or problems found during operation.
