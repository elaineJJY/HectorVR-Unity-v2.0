\selectlanguage{english}
\begin{abstract}
Rescue robotics have been increasingly used in crisis situations, such as exploring areas that are too dangerous for humans, and have become a key research area. A number of studies use \gls{vr} to improve the degree of immersion and situation awareness. However, there remains a need to explore an intuitive, easy-to-use interaction pattern, which increases the efficiency of search and rescue and allows the user to explore the environment intentionally. This thesis presents a preliminary VR-based \gls{hri} system in terms of ground based rescue robots, with the aim to find an ideal interaction pattern. The proposed system offers four different operation modes and corresponding test scenes imitating a post-disaster city. This thesis includes a user study in which four different operation modes are tested in turn. The user study reveals that the ideal interaction pattern should reduce the complexity of the operation as much as possible. Instead of allowing the user to adjust the robot's direction and speed themselves, it is recommended to set a target point and let the robot navigate to the target point automatically. 
\end{abstract}



\selectlanguage{ngerman}
\begin{abstract}
Rettungsroboter werden zunehmend zur Bewältigung von Krisensituationen eingesetzt z.B. dort, wo es für den Menschen zu gefährlich ist. Eine Reihe von Studien nutzt \gls{vr} als Plattform für die Mensch-Roboter-Interaktion, da dies die Immersion erhöhen und das Situationsbewusstsein verbessern kann. Es besteht jedoch nach wie vor Bedarf für eine einfach zu bedienende Interaktionsmethode, um das Personal bei der Steuerung des Roboters zu entlasten und die Effizienz und Effektivität von Such- und Rettungsarbeiten zu erhöhen. In dieser Abschlussarbeit wird ein vorläufiges \gls{vr}-basiertes Mensch-Roboter-Interaktionssystem in Bezug auf bodenbasierte Rettungsroboter vorgestellt. Ziel ist, ein möglichst intuitive Steuerungsmethode mit geringer mentaler Ermüdung zu finden. Das vorgeschlagene System bietet vier verschiedene Betriebsmodi und entsprechende Testszenen. Diese Arbeit beinhaltet eine Nutzerstudie (die vier Betriebsmodi werden nacheinander getestet und verglichen). Das Fazit ist, dass die ideale Interaktionsmethode die Komplexität der Steuerung so weit wie möglich zu reduzieren.
\end{abstract}

\selectlanguage{english}
