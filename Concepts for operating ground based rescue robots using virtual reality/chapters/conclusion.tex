\chapter{Conclusion}
\label{conclusion}

This thesis presents a preliminary VR-based \gls{hri} system in terms of ground based rescue robots. This work aims to find an ideal interaction method or provide a general direction for future development. For this purpose, the proposed system offers four different operation modes and corresponding test scenes imitating a post-disaster city. This thesis shows an overview of the simulated robot, interaction techniques and the construction of the test environment. Eight participants were invited to conduct a user study. Based on the obtained results, it can be concluded that an ideal \gls{vr}-based robotics operation method should eliminate as much complexity as possible. An intelligent obstacle avoidance algorithm is recommended instead of the user operating the robot themselves to steer and move forward. Additional functions, such as monitoring screens, need to be optimized so that they do not complicate the whole interaction process. The system also requires maps to show the user which areas have been detected and where the robot is located.

Future work should focus on the intelligent obstacle avoidance algorithm when using a real robot. The next stage is to develop a live telepresence and teleoperation system with the real robot. Considering that the system proposed in this thesis only simulates the disaster rescue process, the conclusions obtained may be limited. Additional testing and user surveys should be carried out in the future after building a collaborative \gls{vr}-based system with real robots.
