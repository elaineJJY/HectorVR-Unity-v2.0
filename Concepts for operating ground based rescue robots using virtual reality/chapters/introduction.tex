\chapter{Introduction}

In recent years, natural disasters such as earthquakes, tsunamis and potentially nuclear explosives have seriously threatened the safety of human life and property. The number of disasters, as well as their severity and complexity, have gradually increased. The first 72h after a disaster is the golden rescue time, but the unstructured environment of disaster sites makes it difficult for rescuers to work quickly, efficiently and safely.

Rescue robots have the advantages of high mobility. They can work continuously to improve the efficiency of search and rescue. In addition, those robots can map the terrain and detect sound, gas and temperature within the ruins by carrying a variety of sensors.
Rescue robots can assist or replace the rescuers, thus avoiding injuries caused by the secondary collapse and reducing risks for rescuers. 


In fact, rescue robots have already been used in a number of disaster scenarios. 
\gls{crasar} used rescue robots for Urban Search and Rescue task during the World Trade Center collapse in 2001 \cite{Casper:2003tk} and has employed rescue robots at multiple disaster sites to assist in finding survivors, inspecting buildings and scouting the site environment etc. \cite{Murphy:2012th}. Anchor Diver III was utilized as underwater support to search for bodies drowned at sea after the 2011 Tohoku Earthquake and Tsunami \cite{Huang:2011wq}.

Considering the training time and space constraints for rescuers \cite{Murphy:2004wl}, and the goal of efficiency and fluency collaboration \cite{10.1145/1228716.1228718}, the appropriate \gls{hri} approach deserves to be investigated. Some of the existing \gls{hri} solutions include Android software \cite{Sarkar:2017tt} \cite{Faisal:2019uu}, gesture recognition\cite{Sousa:2017tn} \cite{10.1145/2157689.2157818} \cite{Nagi:2014vu}, facial voice recognition \cite{Pourmehr:2013ta}, adopting eye movements \cite{Ma:2015wu}, \gls{ar} \cite{SOARES20151656} and \gls{vr}.

% VR and robot
Among them, \gls{vr} has gained much attention due to its immersion and the interaction method that can be changed virtually. \gls{vr} is no longer a new interaction technique. With the development of technology in recent years, \gls{vr} devices are gradually becoming more accessible to users. With the improvement of hardware devices, the new generation of \gls{vr} headsets has higher resolution and a wider field of view. While \gls{vr} headsets are often considered as entertainment devices, \gls{vr} brings more than that. It plays an important role in many fields such as training, education and medical care.

The use of \gls{vr} in \gls{hrc} also has the potential. In terms of reliability, \gls{vr} is reliable as a novel alternative to \gls{hri}. The interaction tasks that users can accomplish with \gls{vr} do not differ significantly from those using devices in real environments \cite{Villani:2018ub}. In terms of user experience and operational efficiency, \gls{vr} headsets can provide users with stereo viewing cues, which makes collaborative \gls{hri} tasks in certain situations more efficient and performance better \cite{Liu:2017tw}. A novel \gls{vr}-based practical system for immersive robot teleoperation and scene exploration can improve the degree of immersion and situation awareness, thus increasing the accuracy of the robot's navigation and the awareness of objects in the scene. In contrast, this level of immersion and interaction cannot be reached with video-only systems \cite{Stotko:2019ud}.

However, there remains a need to explore \gls{hri} patterns and improve the level of Human-Robot Integration \cite{Wang:2017uy}. Intuitive and easy-to-use interactive patterns can enable the user to explore the environment intentionally and improve the efficiency of search and rescue as much as possible. The appropriate interaction method should cause less mental and physical exhaustion, reducing the need for the user to frequently exit the \gls{vr} environment for rest.

% What I have done (overview)
For this purpose, this thesis presents a preliminary \gls{vr}-based system that simulates the cooperation between ground based rescue robots and humans with four different operation modes and corresponding test scenes, which imitate a post-disaster city. The test scene simulates a robot leveraging \gls{unity} to construct a virtual 3D scene. The proposed system simulates the process that the robot uses \gls{lidar} to detect the surrounding environment. In order to find an interactive approach that is as intuitive and easy-to-use as possible, a user study was performed after the development was completed.


% thesis Architecture
In Chapter \ref{related}, related work involving the integration of \gls{vr} and \gls{hri} is presented.
Chapter \ref{implementation} provides details of the proposed system, including the techniques used for the different interaction modes and the setup for test scenes.
Chapter \ref{evaluate} explains the design and procedure of the user study.
Chapter \ref{result} and \ref{discuss} present the results of the user study and analyze the advantages and disadvantages of the different operation modes and the directions for future work.
Finally, in Chapter \ref{conclusion}, the thesis is concluded.


